\PassOptionsToPackage{unicode=true}{hyperref} % options for packages loaded elsewhere
\PassOptionsToPackage{hyphens}{url}
%
\documentclass[
]{article}
\usepackage{lmodern}
\usepackage{amssymb,amsmath}
\usepackage{ifxetex,ifluatex}
\ifnum 0\ifxetex 1\fi\ifluatex 1\fi=0 % if pdftex
  \usepackage[T1]{fontenc}
  \usepackage[utf8]{inputenc}
  \usepackage{textcomp} % provides euro and other symbols
\else % if luatex or xelatex
  \usepackage{unicode-math}
  \defaultfontfeatures{Scale=MatchLowercase}
  \defaultfontfeatures[\rmfamily]{Ligatures=TeX,Scale=1}
\fi
% use upquote if available, for straight quotes in verbatim environments
\IfFileExists{upquote.sty}{\usepackage{upquote}}{}
\IfFileExists{microtype.sty}{% use microtype if available
  \usepackage[]{microtype}
  \UseMicrotypeSet[protrusion]{basicmath} % disable protrusion for tt fonts
}{}
\makeatletter
\@ifundefined{KOMAClassName}{% if non-KOMA class
  \IfFileExists{parskip.sty}{%
    \usepackage{parskip}
  }{% else
    \setlength{\parindent}{0pt}
    \setlength{\parskip}{6pt plus 2pt minus 1pt}}
}{% if KOMA class
  \KOMAoptions{parskip=half}}
\makeatother
\usepackage{xcolor}
\IfFileExists{xurl.sty}{\usepackage{xurl}}{} % add URL line breaks if available
\IfFileExists{bookmark.sty}{\usepackage{bookmark}}{\usepackage{hyperref}}
\hypersetup{
  pdftitle={BAE 565 Github Week Outline},
  pdfborder={0 0 0},
  breaklinks=true}
\urlstyle{same}  % don't use monospace font for urls
\usepackage[margin=1in]{geometry}
\usepackage{graphicx,grffile}
\makeatletter
\def\maxwidth{\ifdim\Gin@nat@width>\linewidth\linewidth\else\Gin@nat@width\fi}
\def\maxheight{\ifdim\Gin@nat@height>\textheight\textheight\else\Gin@nat@height\fi}
\makeatother
% Scale images if necessary, so that they will not overflow the page
% margins by default, and it is still possible to overwrite the defaults
% using explicit options in \includegraphics[width, height, ...]{}
\setkeys{Gin}{width=\maxwidth,height=\maxheight,keepaspectratio}
\setlength{\emergencystretch}{3em}  % prevent overfull lines
\providecommand{\tightlist}{%
  \setlength{\itemsep}{0pt}\setlength{\parskip}{0pt}}
\setcounter{secnumdepth}{-2}
% Redefines (sub)paragraphs to behave more like sections
\ifx\paragraph\undefined\else
  \let\oldparagraph\paragraph
  \renewcommand{\paragraph}[1]{\oldparagraph{#1}\mbox{}}
\fi
\ifx\subparagraph\undefined\else
  \let\oldsubparagraph\subparagraph
  \renewcommand{\subparagraph}[1]{\oldsubparagraph{#1}\mbox{}}
\fi

% set default figure placement to htbp
\makeatletter
\def\fps@figure{htbp}
\makeatother


\title{BAE 565 Github Week Outline}
\author{}
\date{\vspace{-2.5em}01-15-2021}

\begin{document}
\maketitle

I have worked through the happy git website and have read both the of
the articles you suggested. From this work I have prepared a general
outline for the github/version control week. Please note that this is
just a general idea of what I am thinking. I have not yet prepared any
material towards this goal or developed an excerise that would serve as
homework, though I would want it to deal with forking and cloning a repo
to Github then to a local computer, setting the upstream repo on a local
computer to pull from, making a change on a local computer, pushing your
change back to Github, and then submitting a pull request back to the
upstream repo.

\hypertarget{work-to-be-done-prior-to-12621}{%
\section{Work to be done prior to
1/26/21}\label{work-to-be-done-prior-to-12621}}

\begin{itemize}
\tightlist
\item
  Read sections 1-4 of Excuse Me, Do You Have a Moment to Talk About
  Version Control?
\item
  Read Our path to better science in less time using open data science
  tools
\item
  Complete Section ``1. Installation'' on
  \url{https://happygitwithr.com/}
\end{itemize}

\hypertarget{in-class-work}{%
\section{1/26/21: In class work}\label{in-class-work}}

\begin{itemize}
\tightlist
\item
  0 - 20 min: discussion of article and installation quesitons
\item
  20 - 40 min: Work through section 2 "Connect Git, Github, and RStudio.

  \begin{itemize}
  \tightlist
  \item
    Do this together using slides to allow for quick troubleshooting
  \end{itemize}
\item
  40 - 60 min: Work through section 3 ``Early GitHub Wins''

  \begin{itemize}
  \tightlist
  \item
    Do this together using slides to allow for quick troubleshooting
  \end{itemize}
\item
  60 -75 min: Discussion of Github so far/ start chapter 18 ``Test drive
  in R Markdown''
\end{itemize}

\hypertarget{work-to-be-done-prior-to-12821}{%
\section{Work to be done prior to
1/28/21}\label{work-to-be-done-prior-to-12821}}

\begin{itemize}
\tightlist
\item
  Read sections 8-13 of Excuse Me, Do You Have a Moment to Talk About
  Version Control?
\item
  Complete Chapter 18
\end{itemize}

\hypertarget{section}{%
\section{1/26/21}\label{section}}

\begin{itemize}
\tightlist
\item
  0 - 20 min: dicussin of chapter 18 excericse and use of markdown files
  instead of .docx file
\item
  20 - 60 min: Work through section 4 Daily Workflows
\item
  60 - 75: Present and discuss excerise to be due the following Tuesday
  2/2/21
\end{itemize}

\end{document}
